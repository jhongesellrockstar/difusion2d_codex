\section{Discusi\'on}
Los resultados del Tabla~\ref{tab:benchmark} confirman que existe un compromiso cl\'asico entre tiempo de c\'omputo y precisi\'on. El esquema FTCS alcanza el menor tiempo, pero su error es casi una orden de magnitud mayor que el de la plantilla de trece puntos. El m\'etodo de nueve puntos se ubica en un punto intermedio tanto en costo como en exactitud.

Una comparaci\'on cualitativa con los experimentos de \citet{dehghan2002} muestra la misma tendencia: las variantes enriquecidas con m\'as nodos reducen el error global sin afectar la estabilidad siempre que se respete la condici\'on sobre $r$. En su trabajo, Dehghan report\'o que el esquema \emph{(1,13)} converge m\'as r\'apido que FTCS y el de nueve puntos para un mismo taman\~no de paso, observaci\'on que concuerda con los valores de la Tabla~\ref{tab:benchmark}.

En la pr\'actica, conviene elegir el m\'etodo seg\'un el escenario: FTCS resulta adecuado cuando se prioriza la rapidez o se dispone de pocos recursos; la plantilla de nueve puntos proporciona un balance razonable para mallas de resoluci\'on media; y el esquema de trece puntos es la opci\'on preferible si se busca la mayor precisi\'on posible. Dado que todas las actualizaciones son locales, los tres algoritmos podr\'ian beneficiarse de paralelizaci\'on en CPU o GPU. Incluso podr\'ian emplearse t\'ecnicas modernas, como redes neuronales para acelerar la obtenci\'on de soluciones aproximadas.
