\section{M\'etodos num\'ericos}
En esta secci\'on se describen tres esquemas expl\'icitos empleados para resolver la ecuaci\'on de difusi\'on en dos dimensiones. Sea $u_{i,j}^n$ la aproximaci\'on num\'erica de $u(x_i,y_j,t_n)$ con espaciamientos $\Delta x$, $\Delta y$ y paso temporal $\Delta t$. Denotamos $r_x=\alpha \Delta t/\Delta x^2$ y $r_y=\alpha \Delta t/\Delta y^2$.

\subsection{FTCS}
El esquema \emph{Forward Time Centered Space} actualiza cada nodo usando una plantilla de cinco puntos:
\begin{equation}
  u_{i,j}^{n+1} = u_{i,j}^n + r_x\bigl(u_{i+1,j}^n - 2 u_{i,j}^n + u_{i-1,j}^n\bigr)
  + r_y\bigl(u_{i,j+1}^n - 2 u_{i,j}^n + u_{i,j-1}^n\bigr).
\end{equation}
Su estabilidad exige $r_x + r_y \le 1/2$.

\subsection{Esquema de nueve puntos}
\citet{dehghan2002} propuso extender la plantilla incluyendo las diagonales. Asumiendo $\Delta x = \Delta y = h$ y $r = \alpha \Delta t / h^2$, la f\'ormula queda
\begin{equation}
  u_{i,j}^{n+1} = (1-5 r) u_{i,j}^n + r\,(u_{i+1,j}^n + u_{i-1,j}^n + u_{i,j+1}^n + u_{i,j-1}^n)
  + \tfrac{r}{4} (u_{i+1,j+1}^n + u_{i+1,j-1}^n + u_{i-1,j+1}^n + u_{i-1,j-1}^n).
\end{equation}
Este esquema es estable si $0 < r \le 1/4$.

\subsection{Esquema $(1,13)$}
Otra variante debida a \citet{dehghan2002} utiliza trece puntos, incorporando las cuatro diagonales y los vecinos a dos celdas de distancia:
\begin{equation}
\begin{split}
  u_{i,j}^{n+1} = &\,(1-8 r) u_{i,j}^n + r\,(u_{i+1,j}^n+u_{i-1,j}^n+u_{i,j+1}^n+u_{i,j-1}^n) \\&
  + \tfrac{r}{2}(u_{i+1,j+1}^n+u_{i+1,j-1}^n+u_{i-1,j+1}^n+u_{i-1,j-1}^n)\\&
  + \tfrac{r}{2}(u_{i+2,j}^n+u_{i-2,j}^n+u_{i,j+2}^n+u_{i,j-2}^n).
\end{split}
\end{equation}
Se mantiene la condici\'on de estabilidad $r \le 1/2$.
