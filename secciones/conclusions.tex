\section{Conclusiones}
Se compararon tres m\'etodos expl\'icitos implementados en Python puro. El esquema FTCS fue el m\'as r\'apido, el de nueve puntos brind\'o un compromiso intermedio y el m\'etodo \emph{(1,13)} alcanz\'o la mejor precisi\'on, aunque con mayor costo computacional.

El proyecto constituye un ejercicio id\'oneo para la ense\~nanza de mec\'anica de fluidos computacional, pues permite reproducir todos los experimentos sin depender de software externo.

Como trabajo futuro se propone paralelizar los algoritmos, incorporar condiciones de frontera no homog\'eneas y aplicar las rutinas a problemas de mayor realismo.
